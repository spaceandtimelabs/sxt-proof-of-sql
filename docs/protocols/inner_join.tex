\documentclass[11pt]{article}
\usepackage[T1]{fontenc}
\usepackage[utf8]{inputenc}
\usepackage{lmodern}
\usepackage[margin=1in]{geometry}
\usepackage{graphicx}
\usepackage{amsmath,amssymb}
\usepackage{booktabs}
\usepackage{hyperref}
\usepackage{microtype}
\usepackage{todonotes}
\hypersetup{
  colorlinks=true,
  linkcolor=blue,
  citecolor=blue,
  urlcolor=blue
}
\setlength{\parindent}{0pt}
\setlength{\parskip}{6pt}

\title{Inner Join}
\author{Space and Time Inc}
\date{January 2025}

\begin{document}
\maketitle

\noindent Let $L = (l_{ij})$, $R = (r_{ij})$ be two tables. Join $L$ and $R$ on the columns $C_\ell$ and $C_r$ which have column indexes $j_l = (l'_0,\cdots, l'_{k-1})$ and $j_r = (r'_0,\cdots, r'_{k-1})$ respectively. \\
\noindent Let the prover-provided index columns of $L$ and $R$ be $i_\ell:=\rho_{[0, m_\ell)}$ and $i_r:=\rho_{[0, m_r)}$ respectively. Let $\hat{L} = (L|i_\ell)$ and $\hat{R} = (R|i_r)$ be versions of $L$ and $R$ with the last column the index ones.  The remaining columns of $L$ and $R$ are $A$ and $B$ respectively in their respective original ordering. The inner join $J$ can be expressed as $J = (\bar{C}|\bar{A}|\bar{B})$ where $\bar{C}$ consists of the common join columns in the order of $j_l$, and $\bar{A}$ and $\bar{B}$ are the remaining columns of $J$ from $L$ and $R$ respectively, that is, corresponding to columns of $A$ and $B$.\\
\noindent Let $\tilde{L}$ be a column permutation of $\hat{L}$ by having $C$ first and then other columns of the table, that is, $\tilde{L} = (C_\ell|A|i_\ell)$. Similarly we have $\tilde{R} = (C_r|B|i_r)$. Let $\hat{J}$ be the inner join of $\hat{L}$ and $\hat{R}$ on the columns $C_\ell$ and $C_r$. Hence $\hat{J}=(\bar{C}|\bar{A}|\bar{i_\ell}|\bar{B}|\bar{i_r})$. Let $J_\ell=(\bar{C}|\bar{A}|\bar{i_l})$ and $J_r=(\bar{C}|\bar{B}|\bar{i_r})$. To prove that $J$ is the inner join of $L$ and $R$ we need to prove the following:\\

\section{Summary}
\begin{itemize}
    \item Plan values: $j_\ell$, $j_r$ and number of columns in each of $L$ and $R$
    \item Inputs: $L$, $R$, $w_\ell$, and $w_r$
    \item Outputs: $J$ and $m$
    \item Hints: $m_\ell$, $m_r$, $\bar{i_l}$, $\bar{i_r}$, $U$, $w_l$, $w_r$, and all the internal hints for the gadgets.
\end{itemize}

\section{Details}
\textbf{1. Membership}
\begin{enumerate}
  \item[(a)] $J_\ell$ consists of copies of rows of $\tilde{L}$.
  \item[(b)] $J_r$ consists of copies of rows of $\tilde{R}$.
\end{enumerate}

\noindent Note that with (a) and (b) the join conditions are validated. That is, every row of $J$ has to be a row in the inner join.\\

\textbf{2. Uniqueness of rows of $J$}

We can just focus on the prover-provided index columns $i_\ell$ and $i_r$ or, better yet, let $i = \bar{i_\ell} \cdot 2^{BITS} + \bar{i_r}$. We need to prove that $i$ is strictly increasing (or decreasing).\\
\noindent With (1) and (2), $J$ is a subset of the inner join. That is, if we can prove that the row count of $J$ matches that of the inner join, we will establish that it is exactly the inner join.\\

\textbf{3. Row count}

Let $U$ be the distinct set union of $L'$ and $R'$. We need to prove the following:\\
\begin{enumerate}
\item[(a)] $U$ is strictly increasing (or decreasing).
\item[(b)] $L'$ consist of copies of rows of $U$ with multiplicity vector $w_l$.
\item[(c)] $R'$ consist of copies of rows of $U$ with multiplicity vector $w_r$.
\item[(d)] $w_l \cdot w_r \overset{\Sigma}{=} \chi_m$ with $m$ the number of rows the prover claims $J$ has.
\end{enumerate}

Thus, it is established that $J$ is the inner join of $L$ and $R$ on $L'$ and $R'$.\\

\end{document}

